%%%%%%%%%%%%%%%%%%%%%%%%%%%%%%%%%%%%%%%%%
% Cleese Assignment (For Students)
% LaTeX Template
% Version 2.0 (27/5/2018)
%
% This template originates from:
% http://www.LaTeXTemplates.com
%
% Author:
% Vel (vel@LaTeXTemplates.com)
%
% License:
% CC BY-NC-SA 3.0 (http://creativecommons.org/licenses/by-nc-sa/3.0/)
% 
%%%%%%%%%%%%%%%%%%%%%%%%%%%%%%%%%%%%%%%%%

%----------------------------------------------------------------------------------------
%	PACKAGES AND OTHER DOCUMENT CONFIGURATIONS
%----------------------------------------------------------------------------------------

\documentclass[11pt]{article}

\input{structure.tex} % Include the file specifying the document structure and custom commands

%----------------------------------------------------------------------------------------
%	ASSIGNMENT INFORMATION
%----------------------------------------------------------------------------------------

% Required
\newcommand{\assignmentQuestionName}{Question} % The word to be used as a prefix to question numbers; example alternatives: Problem, Exercise
\newcommand{\assignmentClass}{CNM} % Course/class
\newcommand{\assignmentTitle}{Acceleration of an AI application} % Assignment title or name
\newcommand{\assignmentAuthorName}{Kevin Jordil \& Olivier D'Ancona} % Student name


% Optional (comment lines to remove)
\newcommand{\assignmentClassInstructor}{Professor: Marina Zapater } % Intructor name/time/description
\newcommand{\assignmentDueDate}{Thursday,\ 02\ February\, 2023} % Due date

%----------------------------------------------------------------------------------------

\begin{document}

%----------------------------------------------------------------------------------------
%	TITLE PAGE
%----------------------------------------------------------------------------------------

\maketitle % Print the title page

\thispagestyle{empty} % Suppress headers and footers on the title page

\newpage

\assignmentSection{Stage 1 – Choosing an application}

\section{Multi-Layer Perceptron}

Nous avons $n$ tâches à exécuter sur $m$ machines. Les tâches sont numérotées de $1$ à $n$ et les machines de $1$ à $m$. Les données sont les suivantes:

\begin{itemize}
	\item $p_i$ est le temps d'exécution de la tâche $i$ (processing).
	\item $r_i$ est la date de disponibilité de la tâche $i$ (release).
	\item $d_i$ est la date d'échéance de la tâche $i$ (due).
\end{itemize}

\assignmentSection{Stage 2 – Analysing application bottlenecks}

\assignmentSection{Stage 3 – Acceleration}

\assignmentSection{Stage 4 – Analysis of results}

\end{document}
